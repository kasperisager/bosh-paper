\section{Introduction}
\label{sec:introduction}

A command-line interpreter, often simply referred to as a "shell", is a program that enables a user to interact with an operating system through a command-line interface. The user enters commands into the shell which in return interprets these and executes the corresponding programs if found on the system.

The shell also facilitates communication between different programs through \textit{piping}; the output of one program may be used as the input of another program. \textit{Pipelines} can then be constructed between a series of programs such that each program reads from the program before it and writes to the next.

While piping is one form of \textit{I/O redirection} supported by a shell, input and output may also be redirected to and from files on the file system. As such, a program may read its input from a file and write its output to a file.

Another important aspect of a shell is the ability to execute programs as background processes such that multiple programs may run simultaneously.

\paragraph{Bosh} The goal of our project is to implement a shell, aptly named BOSC Shell or Bosh for short, that supports the above mentioned features. The shell will be written in C and must work independently of any existing shells that might be present on a given system.
